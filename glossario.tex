
%**************************************************************
% Acronimi
%**************************************************************
\renewcommand{\acronymname}{Acronimi e abbreviazioni}

\newacronym[description={\glslink{apig}{Application Program Interface}}]
    {api}{API}{Application Program Interface}

\newacronym[description={\glslink{ui}{UI}}]
{ui}{UI}{User Interface}

\newacronym[description={\glslink{ux}{UX}}]
{uxa}{UX}{User Experience}

\newacronym[description={\glslink{contenutog}{contenuto}}]
{contenuto}{contenuto}{contenuto informativo}

\newacronym[description={\glslink{ci}{CI}}]
{cia}{CI}{Continuos integration}

\newacronym[description={\glslink{cd}{CD}}]
{cda}{CD}{Continuos delivery}

%**************************************************************
% Glossario
%**************************************************************
%\renewcommand{\glossaryname}{Glossario}

\newglossaryentry{appserverg}{ 
    name=\glslink{appserverg}{Application Server},
    text= application server,
    sort= application server,
    description={un \emph{Application Server} è un server che offre le funzionalità necessarie per l'\textit{hosting} di un'applicazione web} 
}

\newglossaryentry{apig}
{
    name=\glslink{apig}{API},
    text=Application Program Interface,
    sort=api,
    description={il termine \emph{Application Programming Interface API} indica un insieme di procedure disponibili al programmatore, di solito raggruppate a formare un set di strumenti specifici per l'espletamento di un determinato compito all'interno di un certo programma. La finalità è ottenere un'astrazione, di solito tra l'hardware e il programmatore o tra software a basso e quello ad alto livello semplificando così il lavoro di programmazione}
}

\newglossaryentry{umlg}
{
    name=\glslink{umlg}{UML},
    text=UML,
    sort=uml,
    description={in ingegneria del software \emph{UML, Unified Modeling Language} (ing. linguaggio di modellazione unificato) è un linguaggio di modellazione e specifica basato sul paradigma object-oriented. L'\emph{UML} svolge un'importantissima funzione di ``lingua franca'' nella comunità della progettazione e programmazione a oggetti. Gran parte della letteratura di settore usa tale linguaggio per descrivere soluzioni analitiche e progettuali in modo sintetico e comprensibile a un vasto pubblico}
}

\newglossaryentry{systemIntegratorg}
{
    name=\glslink{systemIntegratorg}{System integrator},
    text=system Integrator,
    sort=system Integrator,
    description={un \emph{system integrator} si occupa di far dialogare diversi impianti e tecnologie, con lo scopo di creare una nuova struttura che implementi e migliori le funzionalità di quelle di origine}
}

\newglossaryentry{contenutog}
{
    name=\glslink{contenutog}{Contenuto informativo},
    text=contenuti informativi,
    sort=contenuto informativo,
    description={contenuto pubblicitario che può essere visualizzato su sistemi di advertising multimediale situati in luoghi pubblici come, ad esempio, cartelloni o totem pubblicitari}
}

\newglossaryentry{cloudnativog}
{
    name=\glslink{cloudnativog}{Cloud nativo},
    text=cloud nativo,
    sort=cloud nativo,
    description={le tecnologie sviluppate nativamente nel cloud permettono di creare ed eseguire applicazioni in ambienti scalabili, dinamici e moderni. Questo approccio è esemplificato da contenitori, mesh dei servizi, i microservizi, l'infrastruttura non modificabile e le API dichiarative. Tali tecniche consentono di avere sistemi resilienti, gestibili e osservabili, consentendo di apportare modifiche a un elevata frequenza}
}

\newglossaryentry{microservizig}
{
    name=\glslink{microservizig}{Microservizi},
    text=microservizi,
    sort=microservizi,
    description={un architettura a \emph{microservizi} organizza un'applicazione come una raccolta di servizi. In tale arhitettura ciascuna funzione è implementabile in modo indipendente, in questo modo un servizio non funzionante non rischia di compromettere gli altri}
}

\newglossaryentry{iocg}
{
    name=\glslink{iocg}{inversione di controllo},
    text=Inversione di controllo,
    sort=inversione di controllo,
    description={in ingegneria del software l'\emph{inversione di controllo} (o IoC), è un pattern secondo il quale un componente di livello applicativo riceve il controllo da parte di un componente appartenente ad una libreria.}
}

\newglossaryentry{dbmsg}
{
    name=\glslink{dbmsg}{DBMS},
    text=DBMS,
    sort=dbms,
    description={in informatica un \emph{DBMS} (o Database Management System) è un sistema progettato per consentire la creazione, la manipolazione e l'interrogazione a un database}
}

\newglossaryentry{ux}
{
    name=\glslink{ux}{UX},
    text=UX,
    sort=ux,
    description={\emph{UX} (o User Experience) è un termine utilizzato per definire la relazione tra un utente e il prodotto. Un prodotto per aver una buona user experiens deve cercare di comprendere i bisogni dell'utente ed aiutarlo nell'utilizzo del prodotto stesso}
}

\newglossaryentry{ci}
{
    name=\glslink{ci}{CI},
    text=CI,
    sort=ci,
    description={in ingegneria del software la \emph{CI} (o Continuos integration) è una pratica che si applica in contesti in cui lo sviluppo del software avviene attraverso un sistema di controllo versione. Consiste nell'allineamento frequente dagli ambienti di lavoro degli sviluppatori verso l'ambiente condiviso}
}

\newglossaryentry{cd}
{
    name=\glslink{cd}{CD},
    text=CD,
    sort=cd,
    description={in ingegneria del software la \emph{CD} (o Continuos delivery) è una pratica che si applica in contesti in cui lo sviluppo del software avviene attraverso un sistema di controllo versione. Consiste nell'assicurare che il software presente nell'ambiente centrale sia sempre rilasciabile.}
}

\newglossaryentry{ssog}
{
    name=\glslink{ssog}{SSO},
    text=SSO,
    sort=sso,
    description={il \emph{Single Sign On} è una modalità di controllo degli accessi che prevede l'utilizzo di un'unica autenticazione valida su diversi sistemi informatici}
}

\newglossaryentry{mockg}
{
    name=\glslink{mockg}{Mock},
    text=mock,
    sort=mock,
    description={un \emph{mock} è un oggetto simulato che riproduce, in modo controllato, lo stesso comportamento di un oggetto reale}
}

\newglossaryentry{fbwg}
{
    name=\glslink{fbwg}{Feature branch workflow},
    text=Feature branch workflow,
    sort=feature branch workflow,
    description={il \emph{Feature branch workflow} è una flusso di lavoro utilizzato su Git, tale metodo prevede che per ogni feature che si vuole aggiungere al prodotto è necessario creare un nuovo branch. Una volta che il lavoro su una feature è concluso si può procedere con una merge request sul ramo di sviluppo}
}