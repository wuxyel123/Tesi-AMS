
%**************************************************************
% Acronimi
%**************************************************************
\renewcommand{\acronymname}{Acronimi e abbreviazioni}

\newacronym[description={\glslink{apig}{Application Program Interface}}]
    {api}{API}{Application Program Interface}

\newacronym[description={\glslink{umlg}{Unified Modeling Language}}]
    {uml}{UML}{Unified Modeling Language}

%**************************************************************
% Glossario
%**************************************************************
%\renewcommand{\glossaryname}{Glossario}

\newglossaryentry{apig}
{
    name=\glslink{api}{API},
    text=Application Program Interface,
    sort=api,
    description={in informatica con il termine \emph{Application Programming Interface API} (ing. interfaccia di programmazione di un'applicazione) si indica ogni insieme di procedure disponibili al programmatore, di solito raggruppate a formare un set di strumenti specifici per l'espletamento di un determinato compito all'interno di un certo programma. La finalità è ottenere un'astrazione, di solito tra l'hardware e il programmatore o tra software a basso e quello ad alto livello semplificando così il lavoro di programmazione}
}

\newglossaryentry{umlg}
{
    name=\glslink{uml}{UML},
    text=UML,
    sort=uml,
    description={in ingegneria del software \emph{UML, Unified Modeling Language} (ing. linguaggio di modellazione unificato) è un linguaggio di modellazione e specifica basato sul paradigma object-oriented. L'\emph{UML} svolge un'importantissima funzione di ``lingua franca'' nella comunità della progettazione e programmazione a oggetti. Gran parte della letteratura di settore usa tale linguaggio per descrivere soluzioni analitiche e progettuali in modo sintetico e comprensibile a un vasto pubblico}
}

\newglossaryentry{systemIntegratorg}
{
    name=\glslink{System Integrator}{system integrator},
    text=System Integrator,
    sort=system Integrator,
    description=asd
}

\newglossaryentry{contenutog}
{
    name=\glslink{Contenuto informativo}{contenuto informativo},
    text=contenuti informativi,
    sort=contenuto informativo,
    description=asd
}

\newglossaryentry{cloudnativog}
{
    name=\glslink{Cloud nativo}{cloud nativo},
    text=Cloud nativo,
    sort=cloud nativo,
    description=asd
}

\newglossaryentry{appserverg}
{
    name=\glslink{Application server}{application server},
    text=Application server,
    sort=application server,
    description=asd
}

\newglossaryentry{microservizig}
{
    name=\glslink{Microservizi}{microservizi},
    text=Microservizi,
    sort=microservizi,
    description=asd
}

\newglossaryentry{iocg}
{
    name=\glslink{Inversione di controllo}{inversione di controllo},
    text=inversione di controllo,
    sort=inversione di controllo,
    description=asd
}

\newglossaryentry{dbmsg}
{
    name=\glslink{DBMS}{dbms},
    text=DBMS,
    sort=dbms,
    description=asd
}

\newglossaryentry{ui}
{
    name=\glslink{UI}{ui},
    text=UI,
    sort=ui,
    description=asd
}

\newglossaryentry{ux}
{
    name=\glslink{UX}{ux},
    text=UX,
    sort=ux,
    description=asd
}

\newglossaryentry{ci}
{
    name=\glslink{CI}{ci},
    text=CI,
    sort=ci,
    description=asd
}

\newglossaryentry{ssog}
{
    name=\glslink{SSO}{sso},
    text=SSO,
    sort=sso,
    description=asd
}

\newglossaryentry{mockg}
{
    name=\glslink{Mock}{mock},
    text=Mock,
    sort=mock,
    description=asd
}