% !TEX encoding = UTF-8
% !TEX TS-program = pdflatex
% !TEX root = ../tesi.tex

%**************************************************************
\chapter{Obbiettivi e pianificazione}
\label{cap:obbiettivi-pianificazione}
%**************************************************************

\intro{In questo capitolo sono descritti gli obbiettivi dello stage, la pianificazione del lavoro e le aspettative personali.}\\

%**************************************************************
\section{Obbiettivi}
L'obbiettivo di questo stage è la realizzazione di una Proof of Concept dell'applicazione, in cui vengono rese disponibili le funzionalità descritte in \hyperref[sez:scopo]{§1.2}. Per aumentare la produttività e permettere allo studente di conoscere nuove tecnologie viene utilizzato \hyperref[jhi]{JHipster}. Lo studente verrà inserito in un gruppo di lavoro composto da 4 persone in modo da favorire, oltre alla comprensione delle nuove tecnologie, la capacità di lavorare in un team.
Al termine del periodo di stage verrà effettuata una presentazione del prodotto alla direzione dell'azienda.
\\Nel piano di lavoro, documento la cui stesura è avvenuta prima dell'inizio dello stage, sono stati individuati i seguenti obbiettivi suddivisi in obbligatori, desiderabili e opzionali.
\subsection{Obbiettivi obbligatori}
\begin{itemize}
    \item Ob1: Conoscenza del framework Spring e in particolare Spring MVC REST;
    \item Ob2: Interazione e gestione database Oracle;
    \item Ob3: Realizzazione delle funzionalità backend del progetto.
\end{itemize}
\subsection{Obbiettivi desiderabili}
\begin{itemize}
    \item D1: Realizzazione delle funzionalità frontend del progetto;
    \item D2: Conoscenza base sviluppo applicazioni frontend Angular;
    \item D3: Grado di autonomia nel processo di analisi/sviluppo.
\end{itemize}
\subsection{Obbiettivi opzionali}
\begin{itemize}
    \item Op1: Conoscenza base dei strumenti per CI/CD.
\end{itemize}

%**************************************************************
\section{Pianificazione}
Lo stage prevede una durata di 312 ore complessive corrispondenti a 8 ore di lavoro giornalierio per un periodo di circa 8 settimane. L'orario di lavoro è dal Lunedì al Venerdì, dalle ore 9:00 alle 13:00 e dalle ore 14:00 alle 18:00. L'ora tra le 13:00 e le 14:00 è dedicata alla pausa pranzo.
\\La pianificazione redatta per il periodo di stage, divisa per obbiettivi settimanali, è la seguente:
\begin{itemize}
    \item \textbf{prima settimana:}
    \begin{itemize}
        \item studio strumenti di sviluppo (Eclipse, Maven, Git);
        \item analisi dei requisiti.
    \end{itemize}
    \item \textbf{seconda settimana:}
    \begin{itemize}
        \item creazione struttura del database;
        \item gestione database Oracle.
    \end{itemize}
    \item \textbf{terza settimana:}
    \begin{itemize}
        \item studio e utilizzo del framework Spring;
        \item studio e utilizzo di Hibernate;
        \item realizzazione dell'object-relational mapping. 
    \end{itemize}
    \item \textbf{quarta settimana:}
    \begin{itemize}
        \item utilizzo spring MVC e Jackson;
        \item realizzazione dei servizi REST.
    \end{itemize}
    \item \textbf{quinta settimana:}
    \begin{itemize}
        \item studio e utilizzo di elementi avanzati di Oracle;
        \item gestione changeset Liquibase;
        \item sviluppo di altri servizi REST. 
    \end{itemize}
    \item \textbf{sesta settimana:}
    \begin{itemize}
        \item studio di Angular;
        \item sviluppo frontend;
    \end{itemize}
    \item \textbf{settima settimana:}
    \begin{itemize}
        \item continuos integration e continuos delivery;
        \item utilizzo di Sonarqube;
        \item controllo qualità del codice.
    \end{itemize}
    \item \textbf{ottava settimana:}
    \begin{itemize}
        \item gestione cache dell'applicazione;
        \item ottimizzazione;
    \end{itemize}
\end{itemize}

%**************************************************************
\section{Aspettative personali}
Le mie aspettative per quanto riguarda lo stage erano molteplici.
\\Prima tra tutte la possibilità di lavorare in un azienda che produce software in modo da poter sia capire, almeno in parte, come funziona la vita in azienda.
In secondo luogo il mio obbiettivo era quello di imparare nuove tecnologie e le best practice adottate dall'azienda ospitante, anche grazie alla stretta collaborazione con il mio tutor.
Ultima cosa ma non meno importante è la possibilità di farsi conoscere da un azienda in modo da poter pensare ad eventuali collaborazioni future.