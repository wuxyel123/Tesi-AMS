% !TEX encoding = UTF-8
% !TEX TS-program = pdflatex
% !TEX root = ../tesi.tex

%**************************************************************
\chapter{Introduzione}
\label{cap:introduzione}
%**************************************************************

% Introduzione al contesto applicativo.\\

% \noindent Esempio di utilizzo di un termine nel glossario \\
% \gls{api}. \\

% \noindent Esempio di citazione in linea \\
% \cite{site:agile-manifesto}. \\

% \noindent Esempio di citazione nel pie' di pagina \\
%citazione\footcite{womak:lean-thinking} \\

%**************************************************************
\section{L'azienda}

\href{https://scaiitec.it/}{SCAI ITEC\footcite{SCAI ITEC abbrev: ITEC} } è un'azienda italiana appartenente al \href{https://www.grupposcai.it/}{gruppo SCAI}. ITEC si occupa di consulenza, System Integration ed Application management in ambito ICT. L'azienda opera principalmente in settore bancario, assicurativo, industriale e di pubblica amministrazione e servizi. \\Gli elementi chiave del successo e della crescita di SCAI ITEC sono:
\begin{itemize}
    \item vasta e profonda conoscenza delle tecnologie;
    \item grande attenzione per la soddisfazione del cliente;
    \item molta esperienza, maturata nel corso del tempo.
\end{itemize} 
ITEC è oggi una delle maggiori realtà nel nord-est del paese in ambito ICT e si propone come partner per qualsiasi tipo di applicazione, progetto e servizio modellato sulle specifiche necessità del cliente. 
\\Grazie alla consolidata esperienza nel ruolo di \gls{systemIntegratorg}\glsfirstoccur ed alle soluzioni leader di mercato proposte, ITEC è in grado di garantire ai propri clienti risposte rapide, concrete e qualificate in base alle specifiche esigenze di tipo gestionale e applicativo.
\\Ultimo, ma non meno importante, dei motivi per cui l'azienda è all'avanguardia rispetto le nuove tecnologie è il grande investimento di ITEC in ricerca, sviluppo e formazione del personale.

\begin{figure}[h]
    \begin{center}
    \includegraphics[width=0.5\textwidth]{logo-itec}
    \caption{Logo SCAI ITEC}
    \label{fig:figure1}
    \end{center}
\end{figure}


%**************************************************************
\section{Scopo dello stage}
\label{sez:scopo}
L'obiettivo principale dello stage è stato quello di inserire lo studente all'interno di una nuova progettualità, con un particolare focus sulle tematiche legate alle tecnologie multimediali e alla loro distribuzione. Lo studente, affiancato da un IT Architect, ha avuto la possibilità di partecipare al disegno, alla progettazione e realizzazione di una nuova progettualità. L'obbiettivo è stato apprendere le tecnologie e le best practice utilizzate in azienda nel ciclo di vita di un applicativo.
La progettualità vista è volta a creare un software per la gestione di \gls{contenutog}\glsfirstoccur, per la loro creazione, modifica e distribuzione nei vari canali di vendita.
Grazie a questa nuova applicazione si potrà:
\begin{itemize}
    \item creare, modificare eliminare e clonare dei contenuti informativi;
    \item raggruppare i contenuti informativi su segmenti di mercato e distribuirli;
    \item pianificare l’esecuzione e l’aggiornamento dei contenuti informativi sui vari canali di distribuzione;
    \item seguire un processo di Authoring (paradigma Editore, Redattore, Supervisore) nella fase di creazione e distribuzione.
\end{itemize} 
In quanto l'intero progetto è di dimensione non indifferente l'obbiettivo dello stage è stato di sviluppare le funzionalità relative al ruolo di Editore, più nello specifico:
\begin{itemize}
    \item creazione di un contenuto;
    \item modifica di un contenuto;
    \item eliminazione di un contenuto;
    \item preparazione di un contenuto per la distribuzione;
    \item auditing delle azioni effettuate dagli utenti;
    \item documentazione delle funzionalità implementate.
\end{itemize}

%**************************************************************
\section{Tecnologie utilizzate}

\subsection{JHipster}
\label{jhi}
\href{https://www.jhipster.tech/}{JHipster} è una piattaforma di sviluppo, con uno stack tecnologico ben definito, utilizzata per generare, sviluppare e rilasciare, applicazioni e web services all'avanguardia. Supporta molteplici tecnologie per il frontend, tra le quali Angular, React e Vue. Fornisce inoltre supporto per le applicazioni per dispositivi mobili utilizzando Ionic e React Native. Per quanto riguarda il backend, JHipster supporta spring Boot (con l'ausilio di Java o Kotlin), Micronaut, Quarkus, NodeJS e .NET. Per il rilascio sono adottati i principi di \gls{cloudnativog}\glsfirstoccur. Il rilascio è inoltre supportato su AWS, Azure, Cloud Foundry, Google Cloud Platform, Heroku ed OpenShift.
\\L'obbiettivo di JHipster è generare applicazioni web o microservizi all'avanguardia, unendo:
\begin{itemize}
    \item uno stack lato server robusto, ad alte prestazioni e coperto da test;
    \item un interfaccia utente accattivante, moderna e mobile-first usando Angular, React o Vue e Bootstrap per il CSS;
    \item un workflow ben definito per fare la build dell'applicazione con Maven o Gradle;
    \item un architettura a microservizi resiliente,  utilizzando i principi di \gls{cloudnativog};
    \item infrastruttura definita come codice, in modod da rendere la distribuzione su cloud veloce.
\end{itemize}
Nel corso dello stage JHipster è stato utilizzato per generare l'applicazione di base, utilizzata come punto di partenza per lo sviluppo.
\begin{figure}[h]
    \begin{center}
    \includegraphics[width=0.18\textwidth]{logo-jhipster}
    \caption{Logo JHipster}
    \label{fig:figure2}
    \end{center}
\end{figure}

\subsection{Java Enterprise}
\href{https://www.oracle.com/it/java/technologies/java-ee-glance.html}{Java Enterprise}, conosciuto anche come Java EE è un insieme di specifiche che mirano ad estendere Java 8, aggiungendo funzionalità enterprise come elaborazione distribuita e servizi web. Le applicazioni Java EE possono essere eseguite sia come \gls{microservizig}\glsfirstoccur che su \gls{appserverg}\glsfirstoccur. In entrambi i casi vengono gestite: transazionalità, scalabilità, sicurezza e concorrenza.
Nel corso dello stage Java EE è stato utilizzato per la programmazione lato backend.
\begin{figure}[h]
    \begin{center}
    \includegraphics[width=0.28\textwidth]{logo-javaee}
    \caption{Logo Java EE}
    \label{fig:figure3}
    \end{center}
\end{figure}

\subsection{Spring}
\href{https://spring.io/}{Spring} è un framework applicativo open source e un container per l'\gls{iocg}\glsfirstoccur utilizzato dalla piattaforma Java. Le funzionalità di base possono essere usate da una qualsiasi applicazione Java, mentre quelle più avanzate sono disponibili solamente per Java Enterprise.
Il framework Spring ha a se associati vari moduli, nel corso del progetto sono stati utilizzati principalmente:
\begin{itemize}
    \item Spring Boot: utilizzato per creare applicazioni basate su spring eseguibili senza la necessità di configurare un web server;
    \item Spring Data: il cui obbiettivo è di facilitare la gestione e l'interazione di applicazioni Java con un database.
\end{itemize}
\begin{figure}[h]
    \begin{center}
    \includegraphics[width=0.35\textwidth]{logo-spring}
    \caption{Logo Spring}
    \label{fig:figure4}
    \end{center}
\end{figure}

\subsection{Hibernate}
\href{https://hibernate.org/}{Hibernate}Hibernate è un framework per lo sviluppo di applicazioni in Java, utilizzato per gestire e mantenere su un database relazionale un insieme di oggetti Java.
\\Le sue funzionalità principali sono:
\begin{itemize}
    \item mappare oggetti Java come tabelle su database;
    \item convertire i campi dati Java a quelli del \gls{dbmsg}\glsfirstoccur utilizzato;
    \item generare chiamate SQL e convertire la risposta ottenuta in un oggetto Java.
\end{itemize}
Tali funzionalità sono state largamente utilizzate nel corso dello stage.
\begin{figure}[h]
    \begin{center}
    \includegraphics[width=0.35\textwidth]{logo-hibernate}
    \caption{Logo Hibernate}
    \label{fig:figure5}
    \end{center}
\end{figure}

\subsection{REST\footcite{REST: acronimo di Representational State Transfer}}
\href{https://restfulapi.net/}{REST} è un modello architetturale per i sistemi distribuiti. I sistemi rest si basano su HTTP e prevedono una struttura degli URL ben definita, che identifichi univocamente le risorse secondo la convenzione del modello stesso\footcite{Resource Naming: https://restfulapi.net/resource-naming/}. 
In REST per il trasferimento di dati vengono utilizzati i metodi HTTP, più nello specifico:
\begin{itemize}
    \item GET: per il recupero di informazioni;
    \item POST, PUT, PATCH: per l'inserimento di informazioni;
    \item DELETE: per l'eliminazione di informazioni.
\end{itemize}
I principi guida di REST sono:
\begin{itemize}
    \item client-server: separazione dei problemi della UI da quelli di storage dei dati;
    \item statelessness: ogni richiesta deve avere tutte le informazioni necessarie per il suo processamento;
    \item cacheable: i client possono memorizzare in cache le risposte, queste devono essere definite esplicitamente o implicitamente cacheable, in modo da evitare il riutilizzo di dati errati;
    \item uniform interface: utilizzo di un'interfaccia di comunicazione omogenea tra client e server in modo da disaccoppiare e semplificare l'architettura per poterla modificare a blocchi;
    \item layered system:la struttura del sistema può essere composta da strati gerarchici. In questo caso ogni componente non può "vedere" oltre lo strato con cui sta interagendo;
    \item code on demand(opzionale): il codice lato client può essere esteso scaricando ed eseguendo applet o script.
\end{itemize}
Nella definizione di API se queste rispettano tutti i vincoli imposti dall'architettura REST allora possono essere definite RESTful.
\begin{figure}[h]
    \begin{center}
    \includegraphics[width=0.2\textwidth]{logo-rest}
    \caption{Logo REST}
    \label{fig:figure6}
    \end{center}
\end{figure}

\subsection{Oracle}
\label{oracle}
\href{https://www.oracle.com/it/database/}{Oracle database} è un \gls{dbmsg} di tipo relazionale prodotto da \href{https://www.oracle.com/}{Oracle corporation}. 
\\I database oracle sono noti per offrire performance, scalabilità, affidabilità e sicurezza oltre a poter essere utilizzati sia on premise che nel cloud.
\begin{figure}[h]
    \begin{center}
    \includegraphics[width=0.22\textwidth]{logo-oracle}
    \caption{Logo Oracle}
    \label{fig:figure7}
    \end{center}
\end{figure}

\subsection{Liquibase}
\href{https://www.liquibase.org/}{Liquibase} è una libreria open source indipendente dal \gls{dbmsg} utilizzato. Durante il periodo di stage è stata utilizzata per tracciare, gestire e applicare le modifiche allo schema del database.
\begin{figure}[h]
    \begin{center}
    \includegraphics[width=0.23\textwidth]{logo-liquibase}
    \caption{Logo Liquibase}
    \label{fig:figure8}
    \end{center}
\end{figure}
\subsection{Angular}
\href{https://angular.io/}{Angular} è un framework open source per la progettazione e lo sviluppo di applicazioni web.
Le applicazioni angular vengono eseguite interamente a lato client ma grazie alla moltitudine di moduli presenti è possibile integrare un sistema di backend più complesso eseguito lato server.
Nel corso dello stage, il frontend, è stato scritto interamente in Angular.
\begin{figure}[h]
    \begin{center}
    \includegraphics[width=0.23\textwidth]{logo-angular}
    \caption{Logo Angular}
    \label{fig:figure9}
    \end{center}
\end{figure}

%**************************************************************
\newpage
\section{Strumenti di sviluppo}

\subsection{Eclipse}
\label{eclipse}
L'IDE utilizzato per lo sviluppo, previo consiglio del tutor aziendale, è stato \href{https://www.eclipse.org/}{Eclipse}.
Eclipse è un ambiente di sviluppo integrato multipiattaforma e rientra nella categoria di software libero, distribuito secondo i termini della \href{https://www.eclipse.org/legal/epl-2.0/}{Eclipse Public License}.
\begin{figure}[h]
    \begin{center}
    \includegraphics[width=0.35\textwidth]{logo-eclipse}
    \caption{Logo Eclipse}
    \label{fig:figure10}
    \end{center}
\end{figure}

\subsection{Maven}
\label{maven}
\href{https://maven.apache.org/}{Maven} è uno strumento di gestione di progetti software, è basato su un Project Object Model (POM) e può gestire la build, il reporting e la documentazione di un progetto. Nel corso dello stage Maven è stato utilizzato principalmente per automatizzare la build del progetto, sia in sviluppo che in produzione.
\begin{figure}[h]
    \begin{center}
    \includegraphics[width=0.3\textwidth]{logo-maven}
    \caption{Logo Maven}
    \label{fig:figure11}
    \end{center}
\end{figure}

\subsection{Git}
\label{git}
\href{https://git-scm.com/}{Git} è un sistema di controllo di versione distribuito, gratuito ed open source, progettato per gestire progetti di qualsiasi tipo. Nel corso dello stage l'utilizzo di Git è stato affiancato a quello di GitLab, una piattaforma web per la gestione di repository Git.
\begin{figure}[h]
    \begin{center}
    \includegraphics[width=0.2\textwidth]{logo-git}
    \caption{Logo Git}
    \label{fig:figure12}
    \end{center}
\end{figure}
\subsection{SQL Developer}
\href{https://www.oracle.com/database/technologies/appdev/sqldeveloper-landing.html}{SQL Developer} è un ambiente di sviluppo integrato per lavorare con SQL nei database Oracle. Nel corso dello stage è stato utilizzato per gestire e testare il database Oracle usato in produzione.
\begin{figure}[h]
    \begin{center}
    \includegraphics[width=0.18\textwidth]{logo-sqldeveloper}
    \caption{Logo SQL Developer}
    \label{fig:figure13}
    \end{center}
\end{figure}

%**************************************************************
\section{Organizzazione del testo}

\begin{description}
    \item[{\hyperref[cap:obbiettivi-pianificazione]{Il secondo capitolo}}] descrive gli obbietti dello stage, la pianificazione del lavoro effettuata a monte e le aspettative personali riguardanti lo stage;
    
    \item[{\hyperref[cap:metodologia-lavoro]{Il terzo capitolo}}] approfondisce la metodologia di lavoro adottata e i ruoli adottati dal team di sviluppo;
    
    \item[{\hyperref[cap:prodotto-sw]{Il quarto capitolo}}] descrive dettagliatamente le funzionalità del software prodotto e le soluzioni adottate durante la codifica;
    
    \item[{\hyperref[cap:docs-test]{Il quinto capitolo}}] descrive la documentazione prodotta e i test eseguiti;
    
    \item[{\hyperref[cap:considerazioni]{Il sesto capitolo}}] contiene le considerazioni finali riguardanti lo stage e una valutazione personale sul lavoro svolto;

\end{description}

Riguardo la stesura del testo, relativamente al documento sono state adottate le seguenti convenzioni tipografiche:
\begin{itemize}
	\item gli acronimi, le abbreviazioni e i termini ambigui o di uso non comune menzionati vengono definiti nel glossario, situato alla fine del presente documento;
	\item per la prima occorrenza dei termini riportati nel glossario viene utilizzata la seguente nomenclatura: \emph{parola}\glsfirstoccur;
	\item i termini in lingua straniera o facenti parti del gergo tecnico sono evidenziati con il carattere \emph{corsivo}.
\end{itemize}