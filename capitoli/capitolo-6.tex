% !TEX encoding = UTF-8
% !TEX TS-program = pdflatex
% !TEX root = ../tesi.tex

%**************************************************************
\chapter{Considerazioni finali}
\label{cap:considerazioni}
%**************************************************************

\intro{In questo capitolo sono esposte le considerazioni finali sul lavoro svolto corredate da una valutazione personale sullo stage.}\\

%**************************************************************
\section{Soddisfacimento degli obbiettivi}
\subsection{Obbiettivi obbligatori}
\begin{itemize}
    \item Ob1: soddisfatto. Ho imparato informazioni di base e non sull'utilizzo e il funzionamento di Spring. Inoltre ho capito il funzionamento di Spring MVC REST anche grazie al confronto diretto col tutor aziendale;
    \item Ob2: soddisfatto. Il database oracle è stato implementato e ho compreso come interagirci. Ho anche avuto la possibilità di evidenziare le differenze di Oracle rispetto ad altri \gls{dbmsg} di tipo relazionale;
    \item Ob3: Tutte le funzionalità di back-end richieste sono state implementate e testate. Il prodotto finale consente infatti di eseguire tutte le operazioni evidenziate nell'analisi funzionale.
\end{itemize}
\subsection{Obbiettivi desiderabili}
\begin{itemize}
    \item D1: soddisfatto. Il front-end del prodotto è stato implementato e testato. Il front-end è correttamente integrato con il back-end;
    \item D2: soddisfatto. Angular è stato utilizzato per tutto lo sviluppo del front-end. Ho avuto modo di vedere come utilizzare le librerie di Angular e come modificare la presentazione di una pagina;
    \item D3: soddisfatto. Il team di sviluppo è stato ritenuto autonomo e proattivo da parte dei rispettivi tutor aziendali.
\end{itemize}
\subsection{Obbiettivi opzionali}
\begin{itemize}
    \item Op1: parzialmente soddisfatto. Ho avuto modo di comprendere il funzionamento e l'utilità di un processo di CI/CD. Ho capito il funzionamento della pipeline di CI/CD presente nella repository su GitLab tuttavia non ho avuto modo di implementarne una da zero.
\end{itemize}

\section{Conoscenze acquisite}
Di seguito verranno elencate le conoscenze acquisite per ciascuna delle tecnologie utilizzate nel progetto e descritte nelle sezioni \ref{techs} e \ref{strum}.

\subsection{JHipster}
JHipster è stato utilizzato come base di partenza per la creazione dell'applicazione. Nello specifico ho imparato a:
\begin{itemize}
    \item generare applicazioni con JHipster;
    \item importare un modello di database in un applicazione autogenerata;
    \item gestire i problemi legati all'aggiunta di nuove entità nel database;
    \item conoscere l'architettura definita dai programmatori di JHipster.
\end{itemize}

\subsection{Java Enterprise}
Il back-end dell'applicazione è interamente scritto in Java. Nel corso dello stage ho potuto apprendere:
\begin{itemize}
    \item costrutti avanzati di Java;
    \item utilizzo di nuove librerie per lo sviluppo di applicazioni web;
    \item utilizzo di richieste multipart per gestire il caricamento di elementi multimediali;
    \item scrittura di codice Java più leggibile.
\end{itemize}

\subsection{Spring}
L'utilizzo di spring è stato fondamentale per lo sviluppo dell'applicazione. Durante lo sviluppo gli argomenti trattati sono stati i seguenti:
\begin{itemize}
    \item utilizzo di spring MVC per separare la logica dalla presentazione;
    \item utilizzo di Spring data per facilitare l'interazione con il database e garantire la transazionalità delle operazioni;
    \item utilizzo di spring boot in modo da avviare l'applicazione all'interno del container Spring, senza necessità di configurare un web server;
    \item funzionalità core di Spring;
\end{itemize}

\subsection{Hibernate}
Hibernate ha facilitato di molto l'interazione col database, è stato infatti utilizzato per:
\begin{itemize}
    \item mappare le entità definite in java come tabelle su Oracle;
    \item gestire query effettuate ottenendo come risposta un oggetto Java;
    \item gestire la conversione dei campi dati da quelli utilizzati da Java a quelli utilizzati da Oracle.
\end{itemize}

\subsection{Rest}
Le \gls{apig} sono state definite seguendo il modello definito da REST.
Durante la codifica delle \gls{apig} ho avuto modo di:
\begin{itemize}
    \item comprendere meglio le differenze tra i metodi HTTP (GET, PUT, POST, PATCH, DELETE);
    \item definire \gls{apig} RESTful;
    \item evitare e prestare più attenzione ai problemi di sicurezza legati alla logica del codice scritto;
    \item scrivere codice che termina \textit{"gracefully"}, ovvero che anche in caso di errori gravi non causa interruzioni al sistema.
\end{itemize}

\subsection{Oracle}
Durante l'implemetazione di Oracle sono stati trattati diversi argomenti, alcuni non direttamente correlati col database stesso ma comunque degni di nota. Più nello specifico:
\begin{itemize}
    \item aggiunta e utilizzo dei driver Oracle a un progetto Java;
    \item studio delle differenze tra Oracle e altri \gls{dbmsg};
    \item configurazione e utilizzo di una VPN per l'accesso all'intranet aziendale.
\end{itemize}

\subsection{Liquibase}
Liquibase ha permesso di gestire in modo semplice ed efficace le modifiche allo schema del database. Durante lo stage ho imparato a:
\begin{itemize}
    \item modificare manualmente i changelog di Liquibase;
    \item inserire nuove entità e relazioni nello schema;
    \item comprendere e risolvere conflitti nei changelog.
\end{itemize}

\subsection{Angular}
Il front-end di AMS è interamente scritto in Angular. Gli argomenti trattati sono stati:
\begin{itemize}
    \item importazione e utilizzo di moduli Angular;
    \item modifiche all'interfaccia grafica;
    \item inserimento di traduzioni automatiche per gestire l'internazionalizzazione;
    \item integrazione di \gls{apig} RESTful con il front-end.
\end{itemize}

\subsection{Eclipse}
Nonostante avessi già utilizzato Eclipse in precedenza ho comunque acquisito nuove conoscenze di tale strumento, più nello specifico:
\begin{itemize}
    \item gestione di un progetto Maven con Eclipse;
    \item debug dell'applicazione.
\end{itemize}

\subsection{Maven}
La fase di build del progetto è stata gestita interamente con Maven.
Ho avuto modo di apprendere:
\begin{itemize}
    \item il funzionamento del file POM per la gestione delle dipendenze;
    \item la build dell'applicazione tramite Maven;
    \item l'esecuzione dei test automatici tramite Maven;
\end{itemize}

\subsection{Git}
Per gestire il versionamento del software è stata utilizzata una repository su GitLab. Durante lo sviluppo è stato necessario apprendere al meglio l'utilizzo di GIT per gestire il versionamento e i conflitti, più nello specifico:
\begin{itemize}
    \item utilizzo di git da linea di comando;
    \item utilizzo del \gls{fbwg}\glsfirstoccur{};
    \item gestione dei conflitti durante il Merge;
    \item rilascio di una nuova versione del software.
\end{itemize}

\subsection{SQLDeveloper}
Durante il corso del progetto questo strumento è stato utilizzato per:
\begin{itemize}
    \item verificare che i dati fossero inseriti correttamente nel datatabase;
    \item verificare che le query sul database restituissero il risultato aspettato.
\end{itemize}

\section{Valutazione finale}
