% !TEX encoding = UTF-8
% !TEX TS-program = pdflatex
% !TEX root = ../tesi.tex

%**************************************************************
\chapter{Considerazioni finali}
\label{cap:considerazioni}
%**************************************************************

\intro{In questo capitolo sono esposte le considerazioni finali sul lavoro svolto corredate da una valutazione personale sullo stage.}\\

%**************************************************************
\section{Soddisfacimento degli obiettivi}
\subsection{Obiettivi obbligatori}
\begin{itemize}
    \item Ob1: soddisfatto. Ho imparato informazioni di base e avanzate sull'utilizzo e il funzionamento di \textit{Spring}. Inoltre ho compreso più a fondo il funzionamento di \textit{Spring MVC REST} anche mediante il confronto diretto col tutor aziendale;
    \item Ob2: soddisfatto. Il \textit{database oracle} è stato implementato. Ho compreso come interagirci e ho anche avuto la possibilità di evidenziare le differenze di \textit{Oracle} rispetto ad altri \gls{dbmsg} di tipo relazionale;
    \item Ob3: tutte le funzionalità di \textit{back-end} richieste sono state implementate e testate. Il prodotto finale consente, infatti, di eseguire tutte le operazioni evidenziate nell'analisi funzionale.
\end{itemize}
\subsection{Obiettivi desiderabili}
\begin{itemize}
    \item D1: soddisfatto. Il \textit{front-end} del prodotto è stato implementato e testato. Il \textit{front-end} è correttamente integrato con il \textit{back-end};
    \item D2: soddisfatto. \textit{Angular} è stato utilizzato per tutto lo sviluppo del \textit{front-end}. Ho avuto modo di vedere come utilizzare le librerie di \textit{Angular} e come modificare la presentazione di una pagina;
    \item D3: soddisfatto. Il \textit{team} di sviluppo è stato ritenuto autonomo e proattivo da parte dei rispettivi tutor aziendali.
\end{itemize}
\subsection{Obiettivi opzionali}
\begin{itemize}
    \item Op1: parzialmente soddisfatto. Ho avuto modo di comprendere il funzionamento e l'utilità di un processo di \gls{ci}/\gls{cd}. Ho appreso il funzionamento della \textit{pipeline} di \gls{ci}/\gls{cd} presente nella \textit{repository} su \textit{GitLab}, tuttavia non ho avuto l'occasione di implementarne una da zero.
\end{itemize}

\section{Conoscenze acquisite}
Di seguito verranno elencate le conoscenze acquisite per ciascuna delle tecnologie utilizzate nel progetto e descritte nelle sezioni \ref{techs} e \ref{strum}.

\subsection{JHipster}
\textit{JHipster} è stato utilizzato come base di partenza per la creazione dell'applicazione. Nello specifico ho imparato a:
\begin{itemize}
    \item generare applicazioni con \textit{JHipster};
    \item importare un modello di database in un'applicazione autogenerata;
    \item gestire i problemi legati all'aggiunta di nuove entità nel \textit{database};
    \item conoscere l'architettura definita dai programmatori di \textit{JHipster}.
\end{itemize}

\subsection{Java Enterprise}
Il \textit{back-end} dell'applicazione è interamente scritto in \textit{Java}. Nel corso dello stage ho potuto apprendere:
\begin{itemize}
    \item costrutti avanzati di \textit{Java};
    \item utilizzo di nuove librerie per lo sviluppo di applicazioni web;
    \item utilizzo di richieste \textit{multipart} per gestire il caricamento di elementi multimediali;
    \item scrittura di codice \textit{Java} più leggibile.
\end{itemize}

\subsection{Spring}
L'utilizzo di \textit{Spring} è stato fondamentale per lo sviluppo dell'applicazione. Durante lo sviluppo gli argomenti trattati sono stati i seguenti:
\begin{itemize}
    \item utilizzo di \textit{Spring} \textit{MVC} per separare la logica dalla presentazione;
    \item utilizzo di \textit{Spring} data per facilitare l'interazione con il database e per garantire la transazionalità delle operazioni;
    \item utilizzo di \textit{Spring} \textit{boot} in modo da avviare l'applicazione all'interno del container \textit{Spring} senza necessità di configurare un \textit{web server};
    \item funzionalità core di \textit{Spring};
\end{itemize}

\subsection{Hibernate}
\textit{Hibernate} ha facilitato di molto l'interazione col \textit{database}. Esso è stato infatti utilizzato per:
\begin{itemize}
    \item mappare le entità definite in \textit{Java} come tabelle su \textit{Oracle};
    \item gestire \textit{query} effettuate ottenendo come risposta un oggetto \textit{Java};
    \item gestire la conversione dei campi dati da quelli utilizzati da \textit{Java} a quelli utilizzati da \textit{Oracle}.
\end{itemize}

\subsection{Rest}
Le \gls{apig} sono state definite seguendo il modello definito da \textit{REST}.
Durante la codifica delle \gls{apig} ho avuto modo di:
\begin{itemize}
    \item comprendere meglio le differenze tra i metodi \textit{HTTP (GET, PUT, POST, PATCH, DELETE)};
    \item definire \gls{apig} \textit{RESTful};
    \item evitare e prestare più attenzione ai problemi di sicurezza legati alla logica del codice scritto;
    \item scrivere codice che termina \textit{"gracefully"}, ovvero che anche in caso di errori gravi non causa interruzioni al sistema.
\end{itemize}

\subsection{Oracle}
Durante l'implemetazione di \textit{Oracle} sono stati trattati diversi argomenti, alcuni non direttamente correlati col \textit{database} stesso ma comunque degni di nota. Più nello specifico:
\begin{itemize}
    \item aggiunta e utilizzo dei driver \textit{Oracle} a un progetto \textit{Java};
    \item studio delle differenze tra \textit{Oracle} e altri \gls{dbmsg};
    \item configurazione e utilizzo di una \textit{VPN} per l'accesso all'intranet aziendale.
\end{itemize}

\subsection{Liquibase}
\textit{Liquibase} ha permesso di gestire in modo semplice ed efficace le modifiche allo schema del \textit{database}. Durante lo stage ho imparato a:
\begin{itemize}
    \item modificare manualmente i \textit{changelog} di \textit{Liquibase};
    \item inserire nuove entità e relazioni nello schema;
    \item comprendere e risolvere conflitti nei \textit{changelog}.
\end{itemize}

\subsection{Angular}
Il \textit{front-end} di AMS è interamente scritto in \textit{Angular}. Gli argomenti trattati sono stati:
\begin{itemize}
    \item importazione e utilizzo di moduli \textit{Angular};
    \item modifiche all'interfaccia grafica;
    \item inserimento di traduzioni automatiche per gestire l'internazionalizzazione;
    \item integrazione di \gls{apig} RESTful con il \textit{front-end}.
\end{itemize}

\subsection{Eclipse}
Nonostante avessi già utilizzato \textit{Eclipse} in precedenza, ho acquisito nuove conoscenze di tale strumento, più nello specifico:
\begin{itemize}
    \item gestione di un progetto \textit{Maven} con \textit{Eclipse};
    \item \textit{debug} dell'applicazione.
\end{itemize}

\subsection{Maven}
La fase di \textit{build} del progetto è stata gestita interamente con \textit{Maven}.
Ho avuto modo di apprendere:
\begin{itemize}
    \item il funzionamento del \textit{file POM} per la gestione delle dipendenze;
    \item la \textit{build} dell'applicazione tramite \textit{Maven};
    \item l'esecuzione dei \textit{test} automatici tramite \textit{Maven};
\end{itemize}

\subsection{Git}
Per gestire il versionamento del \textit{software} è stata utilizzata una \textit{repository} su \textit{GitLab}. Durante lo sviluppo è stato necessario apprendere al meglio l'utilizzo di \textit{GIT} per gestire il versionamento e i conflitti, in particolare:
\begin{itemize}
    \item utilizzo di \textit{GIT} da linea di comando;
    \item utilizzo del \gls{fbwg}\glsfirstoccur{};
    \item gestione dei conflitti durante il \textit{merge};
    \item rilascio di una nuova versione del \textit{software}.
\end{itemize}

\subsection{SQLDeveloper}
Durante il corso del progetto questo strumento è stato utilizzato per:
\begin{itemize}
    \item verificare che i dati fossero inseriti correttamente nel \textit{database};
    \item verificare che le \textit{query} sul \textit{database} restituissero il risultato aspettato.
\end{itemize}

\section{Valutazione finale}
Sono molto soddisfatto dell'esperienza fatta durante il periodo di stage. Ho avuto la possibilità di lavorare all'interno di un'azienda assieme ad un \textit{team} di lavoro composto da un altro stagista e dai due tutor. Questo ambiente ha favorito l'apprendimento di competenze tecniche e, soprattutto, ha migliorato la mia capacità di lavorare in gruppo all'interno di un contesto lavorativo. Per quanto riguarda il prodotto, tutti gli obiettivi prefissati sono stati completati e alla fine dello stage è stata fatta una presentazione del prodotto al direttore. La presentazione è andata a buon fine: tutte le funzionalità richieste sono state mostrate. Al termine della stessa ci sono state fatte alcune domande mediante una modalità peculiare a quella utilizzata nei colloqui. Qualche giorno dopo sono stato nuovamente contattato dall'azienda che mi ha presentato una proposta di assunzione, invitandomi prima a terminare gli studi. Ciò mi ha reso molto fiero del lavoro svolto. Reputo quindi questa esperienza completamente positiva.