% !TEX encoding = UTF-8
% !TEX TS-program = pdflatex
% !TEX root = ../tesi.tex

%**************************************************************
\chapter{Introduzione}
\label{cap:introduzione}
%**************************************************************

% Introduzione al contesto applicativo.\\

% \noindent Esempio di utilizzo di un termine nel glossario \\
% \gls{api}. \\

% \noindent Esempio di citazione in linea \\
% \cite{site:agile-manifesto}. \\

% \noindent Esempio di citazione nel pie' di pagina \\
%citazione\footcite{womak:lean-thinking} \\

%**************************************************************
\section{L'azienda}

\href{https://scaiitec.it/}{SCAI ITEC} è un'azienda italiana appartenente al \href{https://www.grupposcai.it/}{gruppo SCAI}. ITEC si occupa di consulenza, \textit{System Integration} e \textit{Application management} in ambito ICT. L'azienda opera principalmente in settore bancario, assicurativo, industriale e di pubblica amministrazione e servizi. \\Gli elementi chiave del successo e della crescita di SCAI ITEC sono:
\begin{itemize}
    \item vasta e profonda conoscenza delle tecnologie;
    \item grande attenzione per la soddisfazione del cliente;
    \item molta esperienza, maturata nel corso del tempo.
\end{itemize} 
ITEC è oggi una delle maggiori realtà nel nord-est del paese in ambito ICT e si propone come partner per qualsiasi tipo di applicazione, progetto e servizio modellato sulle specifiche necessità del cliente. 
\\Grazie alla consolidata esperienza nel ruolo di \textit{\gls{systemIntegratorg}\glsfirstoccur{}} e alle soluzioni \textit{leader} di mercato proposte, ITEC è in grado di garantire ai propri clienti risposte rapide, concrete e qualificate in base alle specifiche esigenze di tipo gestionale e applicativo.
\\Ultimo, ma non meno importante, dei motivi per cui l'azienda è all'avanguardia rispetto le nuove tecnologie è il grande investimento di ITEC in ricerca, sviluppo e formazione del personale.

\begin{figure}[h]
    \begin{center}
    \includegraphics[width=0.5\textwidth]{logo-itec}
    \caption{Logo SCAI ITEC}
    \label{fig:figure1}
    \end{center}
\end{figure}


%**************************************************************
\section{Scopo dello stage}
\label{sez:scopo}
L'obiettivo principale dello stage è stato quello di inserire lo studente all'interno di una nuova progettualità, con un particolare \textit{focus} sulle tematiche legate alle tecnologie multimediali e alla loro distribuzione. Lo studente, affiancato da un \textit{IT Architect}, ha avuto la possibilità di partecipare al disegno, alla progettazione e realizzazione di una nuova progettualità. L'obbiettivo è stato apprendere le tecnologie e le \textit{best practice} utilizzate in azienda, durante il ciclo di vita di un applicativo.
La progettualità vista è volta a creare un \textit{software} per la gestione di \gls{contenutog}\glsfirstoccur{}, per la loro creazione, modifica e distribuzione nei vari canali di vendita.
Grazie a questa nuova applicazione si potrà:
\begin{itemize}
    \item creare, modificare eliminare e clonare dei \gls{contenutog};
    \item raggruppare i \gls{contenutog} su segmenti di mercato e distribuirli;
    \item pianificare la visualizzazione e l’aggiornamento dei \gls{contenutog} sui vari canali di distribuzione;
    \item seguire un processo di \textit{Authoring} (paradigma Editore, Redattore, Supervisore) nella fase di creazione e distribuzione.
\end{itemize} 
In quanto l'intero progetto è di dimensione non indifferente l'obbiettivo dello stage è stato di sviluppare le funzionalità relative al ruolo di editore, più nello specifico:
\begin{itemize}
    \item creazione di un contenuto;
    \item modifica di un contenuto;
    \item eliminazione di un contenuto;
    \item preparazione di un contenuto per la distribuzione;
    \item \textit{auditing} delle azioni effettuate dagli utenti;
    \item documentazione delle funzionalità implementate.
\end{itemize}

\section{Organizzazione del testo}

\begin{description}
    \item[{\hyperref[cap:obbiettivi-pianificazione]{Il secondo capitolo}}] descrive gli obbietti dello stage, la pianificazione del lavoro effettuata a monte e le aspettative personali riguardanti lo stage;
    
    \item[{\hyperref[cap:tecnologie]{Il terzo capitolo}}] descrive le tecnologie e gli strumenti di sviluppo utilizzati durante il corso dello stage;
    
    \item[{\hyperref[cap:metodologia-lavoro]{Il quarto capitolo}}] approfondisce la metodologia di lavoro e i ruoli adottati dal team di sviluppo;
    
    \item[{\hyperref[cap:analisi]{Il quinto capitolo}}] descrive l'analisi, la progettazione iniziale, e le soluzioni adottate durante la codifica;
    
    \item[{\hyperref[cap:prodotto]{Il sesto capitolo}}] descrive le funzionalità del prodotto finale, la documentazione e i test eseguiti;
    
    \item[{\hyperref[cap:considerazioni]{Il settimo capitolo}}] contiene le considerazioni finali riguardanti lo stage e una valutazione personale sul lavoro svolto;

\end{description}

Riguardo la stesura del testo, relativamente al documento sono state adottate le seguenti convenzioni tipografiche:
\begin{itemize}
	\item gli acronimi, le abbreviazioni e i termini ambigui o di uso non comune menzionati vengono definiti nel glossario, situato alla fine del presente documento;
	\item per la prima occorrenza dei termini riportati nel glossario viene utilizzata la seguente nomenclatura: \emph{parola}\glsfirstoccur{};
	\item i termini in lingua straniera o facenti parti del gergo tecnico sono evidenziati con il carattere \emph{corsivo}.
\end{itemize}