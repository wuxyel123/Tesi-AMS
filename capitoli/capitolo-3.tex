% !TEX encoding = UTF-8
% !TEX TS-program = pdflatex
% !TEX root = ../tesi.tex

%**************************************************************
\chapter{Metodologia di sviluppo}
\label{cap:metodologia-lavoro}
%**************************************************************

\intro{In questo capitolo viene descritta la metodologia di lavoro e i ruoli adottati dal team di sviluppo.}\\

%**************************************************************
\section{Scrum}
\href{https://www.scrum.org/about}{Scrum} è un framework agile per lo sviluppo, consegna e manutenzione di prodotti software e non.
Scrum è progettato per l'utilizzo in team di dimensione ridotta. 
Di seguito vengono descritti ruoli, fasi e artefatti del framework.

\subsection{Ruoli}

\subsubsection{Scrum master}
Lo Scrum Master aiuta il team di sviluppo ad apprendere e applicare Scrum per conseguire valore di business. Lo Scrum Master fa tutto ciò che è in suo potere per aiutare il Team, il Product Owner e l'organizzazione ad avere successo. Lo ScrumMaster non è il manager dei membri del Team, né è un project manager, team leader, o rappresentante del team.
Lo scopo dello Scrum Master è:
\begin{itemize}
    \item aiutare a rimuovere gli ostacoli durante lo sviluppo;
    \item evitare interferenze esterne;
    \item aiutare il Team ad adottare al meglio le pratiche di sviluppo agile;
    \item fare in modo che tutti applichino Scrum nel miglior modo possibile.
\end{itemize}

\subsubsection{Product Owner}
Il Product Owner ha la responsabilità di massimizzare il ritorno sugli investimenti (ROI), di identificare le caratteristiche del prodotto, traducendole in una lista di priorità, di decidere cosa dovrebbe andare in cima alla lista per il prossimo Sprint, e di riassegnare le priorità, aggiornandole con continuità. Il Product owner detiene la responsabilità di profitto del prodotto, se questo è commerciale. In Agile il Product owner rappresenta il cliente e nell'applicazione di Scrum può e deve:
\begin{itemize}
    \item definire il Product Backlog, le user stories e gli acceptance criteria;
    \item definire le priorità nel Product Backlog e la data di rilascio del prodotto;
    \item accettare o rifiutare quanto sviluppato;
    \item cancellare lo sprint se risulta fallimentare o poco utile.
\end{itemize}


\subsubsection{Team di sviluppo}
Il team di sviluppo è composto da un insieme di persone, in genere meno di 10, e si occupa di sviluppare quanto definito dal product owner. Il team Scrum deve essere "cross-funzionale", ovvero includere tutte le competenze necessarie allo sviluppo del prodotto. I membri del team devono essere proattivi e aperti allo studio di tecnologie che vanno oltre le loro competenze.
Il team di sviluppo:
\begin{itemize}
    \item costruisce il prodotto definito dal Product owner;
    \item possiede tutte le conoscenze per ottenere un prodotto potenzialmente rilasciabile alla fine di ogni sprint;
    \item è auto organizzato, con un alto grado di autonomia e responsabilità;
    \item decide quanti e quali elementi del Product backlog sviluppare;
    \item ha la responsabilità di sviluppo, test e rilascio del prodotto;
    \item non possiede un team leader, in quanto in Scrum nel team di sviluppo sono considerati tutti di pari livello.
\end{itemize}

\subsection{Artefatti}

\subsubsection{Product backlog}

\subsubsection{Definition of done}

\subsection{Fasi}

\subsubsection{Sprint planning}

\subsubsection{Daily scrum}

\subsubsection{Sprint review}

\subsubsection{Sprint retrospective}

%**************************************************************
\section{Scrum applicato ad Advertising Management System}