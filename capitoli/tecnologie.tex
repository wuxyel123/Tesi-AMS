% !TEX encoding = UTF-8
% !TEX TS-program = pdflatex
% !TEX root = ../tesi.tex

%**************************************************************
\chapter{Tecnologie e strumenti di sviluppo}
\label{cap:tecnologie}

\intro{In questo capitolo sono descritte le tecnologie e gli strumenti di sviluppo utilizzati durante il corso dello stage.}\\

\section{Tecnologie}
\label{techs}

\subsection{JHipster}
\label{jhi}
\href{https://www.jhipster.tech/}{JHipster} è una piattaforma di sviluppo, con uno \textit{stack} tecnologico ben definito, utilizzata per generare, sviluppare e rilasciare, applicazioni e \textit{web services} all'avanguardia. Supporta molteplici tecnologie per il \textit{frontend}, tra le quali \textit{Angular}, \textit{React} e \textit{Vue}. Fornisce inoltre supporto per le applicazioni per dispositivi mobili utilizzando \textit{Ionic} e \textit{React Native}. Per quanto riguarda il \textit{backend}, JHipster supporta \textit{spring Boot} (con l'ausilio di \textit{Java} o \textit{Kotlin}), \textit{Micronaut}, \textit{Quarkus}, \textit{NodeJS} e \textit{.NET}. Per il rilascio sono adottati i principi di \gls{cloudnativog}\glsfirstoccur{}. Il rilascio è supportato su \textit{AWS}, \textit{Azure}, \textit{Cloud Foundry}, \textit{Google Cloud Platform}, \textit{Heroku} e \textit{OpenShift}.
\\L'obiettivo di JHipster è generare applicazioni web o microservizi all'avanguardia, unendo:
\begin{itemize}
    \item uno \textit{stack} lato server robusto, ad alte prestazioni e coperto da test;
    \item un'interfaccia utente accattivante, moderna e \textit{mobile-first} usando \textit{Angular}, \textit{React} o \textit{Vue} e \textit{Bootstrap} per il \textit{CSS};
    \item un \textit{workflow} ben definito per fare la \textit{build} dell'applicazione con \textit{Maven} o \textit{Gradle};
    \item un'architettura a microservizi resiliente,  utilizzando i principi di \gls{cloudnativog};
    \item infrastruttura definita come codice, in modo da rendere la distribuzione su \textit{cloud} veloce.
\end{itemize}
\newpage
Nel corso dello stage JHipster è stato utilizzato per generare l'applicazione di base, utilizzata come punto di partenza per lo sviluppo.
\begin{figure}[h]
    \begin{center}
    \includegraphics[width=0.18\textwidth]{logo-jhipster}
    \caption{Logo JHipster}
    \label{fig:figure2}
    \end{center}
\end{figure}

\subsection{Java Enterprise}
\href{https://www.oracle.com/it/java/technologies/java-ee-glance.html}{Java Enterprise}, conosciuto anche come \textit{Java EE} è un insieme di specifiche che mirano a estendere Java 8, aggiungendo funzionalità \textit{enterprise} come elaborazione distribuita e servizi web. Le applicazioni \textit{Java EE} possono essere eseguite sia come \gls{microservizig}\glsfirstoccur{} che su \gls{appserverg}\glsfirstoccur{}. In entrambi i casi vengono gestite: transazionalità, scalabilità, sicurezza e concorrenza.
Nel corso dello stage \textit{Java EE} è stato utilizzato per la programmazione lato \textit{backend}.
\begin{figure}[h]
    \begin{center}
    \includegraphics[width=0.28\textwidth]{logo-javaee}
    \caption{Logo Java EE}
    \label{fig:figure3}
    \end{center}
\end{figure}

\subsection{Spring}
\href{https://spring.io/}{Spring} è un \textit{framework} applicativo \textit{open source} e un \textit{container} per l'\gls{iocg}\glsfirstoccur{} utilizzato dalla piattaforma \textit{Java}. Le funzionalità di base possono essere usate da una qualsiasi applicazione \textit{Java}, mentre quelle più avanzate sono disponibili solamente per \textit{Java Enterprise}.
Il \textit{framework Spring} ha a se associati vari moduli, nel corso del progetto sono stati utilizzati principalmente:
\begin{itemize}
    \item \textit{Spring Boot}: utilizzato per creare applicazioni basate su \textit{}, eseguibili senza la necessità di configurare un \textit{web server};
    \item \textit{Spring Data}: il cui obiettivo è di facilitare la gestione e l'interazione di applicazioni \textit{Java} con un \textit{database}.
\end{itemize}
\begin{figure}[h]
    \begin{center}
    \includegraphics[width=0.35\textwidth]{logo-spring}
    \caption{Logo Spring}
    \label{fig:figure4}
    \end{center}
\end{figure}

\subsection{Hibernate}
\href{https://hibernate.org/}{Hibernate} è un \textit{framework} per lo sviluppo di applicazioni in Java, utilizzato per gestire e mantenere su un \textit{database} relazionale un insieme di oggetti \textit{Java}.
\\Le sue funzionalità principali sono:
\begin{itemize}
    \item mappare oggetti \textit{Java} come tabelle su \textit{database};
    \item convertire i campi dati \textit{Java} a quelli del \gls{dbmsg}\glsfirstoccur{}{} utilizzato;
    \item generare chiamate \textit{SQL} e convertire la risposta ottenuta in un oggetto \textit{Java}.
\end{itemize}
Tali funzionalità sono state largamente utilizzate nel corso dello stage.
\begin{figure}[h]
    \begin{center}
    \includegraphics[width=0.35\textwidth]{logo-hibernate}
    \caption{Logo Hibernate}
    \label{fig:figure5}
    \end{center}
\end{figure}

\subsection{REST}
\href{https://restfulapi.net/}{REST} è un modello architetturale per i sistemi distribuiti. I sistemi \textit{REST} si basano su \textit{HTTP} e prevedono una struttura degli \textit{URL} ben definita, che identifichi univocamente le risorse secondo la convenzione del modello stesso\footcite{Resource Naming: https://restfulapi.net/resource-naming/}. 
In \textit{REST} per il trasferimento di dati vengono utilizzati i metodi \textit{HTTP}, più nello specifico:
\begin{itemize}
    \item \textit{GET}: per il recupero di informazioni;
    \item \textit{POST, PUT, PATCH}: per l'inserimento di informazioni;
    \item \textit{DELETE}: per l'eliminazione di informazioni.
\end{itemize}
I principi guida di \textit{REST} sono:
\begin{itemize}
    \item \textit{client-server}: separazione dei problemi della \gls{ui}\glsfirstoccur da quelli di storage dei dati;
    \item \textit{statelessness}: ogni richiesta deve avere tutte le informazioni necessarie per il suo processamento;
    \item \textit{cacheable}: i client possono memorizzare in cache le risposte, queste devono essere definite esplicitamente o implicitamente \textit{cacheable}, in modo da evitare il riutilizzo di dati errati;
    \item \textit{uniform interface}: utilizzo di un'interfaccia di comunicazione omogenea tra \textit{client} e \textit{server} in modo da disaccoppiare e semplificare l'architettura per poterla modificare a blocchi;
    \item \textit{layered system}: la struttura del sistema può essere composta da strati gerarchici. In questo caso ogni componente non può "vedere" oltre lo strato con cui sta interagendo;
    \item \textit{code on demand(opzionale)}: il codice lato \textit{client} può essere esteso scaricando ed eseguendo \textit{applet} o \textit{script}.
\end{itemize}
Nella definizione di \gls{apig} se queste rispettano tutti i vincoli imposti dall'architettura \textit{REST} allora possono essere definite \textit{RESTful}.
\begin{figure}[h]
    \begin{center}
    \includegraphics[width=0.2\textwidth]{logo-rest}
    \caption{Logo REST}
    \label{fig:figure6}
    \end{center}
\end{figure}

\subsection{Oracle}
\label{oracle}
\href{https://www.oracle.com/it/database/}{Oracle database} è un \gls{dbmsg} di tipo relazionale prodotto da \textit{\href{https://www.oracle.com/}{Oracle corporation}}. 
\\I \textit{database Oracle} sono noti per offrire \textit{performance}, scalabilità, affidabilità e sicurezza oltre a poter essere utilizzati sia \textit{on premise} che nel \textit{cloud}.
\begin{figure}[h]
    \begin{center}
    \includegraphics[width=0.22\textwidth]{logo-oracle}
    \caption{Logo Oracle}
    \label{fig:figure7}
    \end{center}
\end{figure}

\subsection{Liquibase}
\href{https://www.liquibase.org/}{Liquibase} è una libreria \textit{open source} indipendente dal \gls{dbmsg} utilizzato. Durante il periodo di stage è stata utilizzata per tracciare, gestire e applicare le modifiche allo schema del \textit{database}.
\begin{figure}[h]
    \begin{center}
    \includegraphics[width=0.23\textwidth]{logo-liquibase}
    \caption{Logo Liquibase}
    \label{fig:figure8}
    \end{center}
\end{figure}
\newpage
\subsection{Angular}
\href{https://angular.io/}{Angular} è un \textit{framework open source} per la progettazione e lo sviluppo di applicazioni web.
Le applicazioni \textit{Angular} vengono eseguite interamente lato \textit{client} ma grazie alla moltitudine di moduli presenti è possibile integrare un sistema di \textit{backend} più complesso eseguito lato \textit{server}.
Nel corso dello stage, il \textit{frontend}, è stato scritto interamente in \textit{Angular}.
\begin{figure}[h]
    \begin{center}
    \includegraphics[width=0.23\textwidth]{logo-angular}
    \caption{Logo Angular}
    \label{fig:figure9}
    \end{center}
\end{figure}

%**************************************************************
\newpage
\section{Strumenti di sviluppo}
\label{strum}

\subsection{Eclipse}
\label{eclipse}
L'\textit{IDE} utilizzato per lo sviluppo, previo consiglio del \textit{tutor aziendale}, è stato \textit{\href{https://www.eclipse.org/}{Eclipse}}.
\textit{Eclipse} è un ambiente di sviluppo integrato multipiattaforma e rientra nella categoria di \textit{software} libero, distribuito secondo i termini della \textit{\href{https://www.eclipse.org/legal/epl-2.0/}{Eclipse Public License}}.
\begin{figure}[h]
    \begin{center}
    \includegraphics[width=0.35\textwidth]{logo-eclipse}
    \caption{Logo Eclipse}
    \label{fig:figure10}
    \end{center}
\end{figure}

\subsection{Maven}
\label{maven}
\href{https://maven.apache.org/}{Maven} è uno strumento di gestione di progetti \textit{software}, è basato su un \textit{Project Object Model (POM)} e può gestire la \textit{build}, il \textit{reporting} e la documentazione di un progetto. Nel corso dello stage \textit{Maven} è stato utilizzato principalmente per automatizzare la \textit{build} del progetto, sia in sviluppo che in produzione.
\begin{figure}[h]
    \begin{center}
    \includegraphics[width=0.3\textwidth]{logo-maven}
    \caption{Logo Maven}
    \label{fig:figure11}
    \end{center}
\end{figure}

\subsection{Git}
\label{git}
\href{https://git-scm.com/}{Git} è un sistema di controllo di versione distribuito, gratuito e \textit{open source}, progettato per gestire progetti di qualsiasi tipo. Nel corso dello stage l'utilizzo di \textit{Git} è stato affiancato a quello di \textit{GitLab}, una piattaforma web per la gestione di \textit{repository Git}.
\begin{figure}[h]
    \begin{center}
    \includegraphics[width=0.2\textwidth]{logo-git}
    \caption{Logo Git}
    \label{fig:figure12}
    \end{center}
\end{figure}
\subsection{SQL Developer}
\href{https://www.oracle.com/database/technologies/appdev/sqldeveloper-landing.html}{SQL Developer} è un ambiente di sviluppo integrato per lavorare con \textit{SQL} nei \textit{database Oracle}. Nel corso dello stage è stato utilizzato per gestire e testare il \textit{database Oracle} usato in produzione.
\begin{figure}[h]
    \begin{center}
    \includegraphics[width=0.18\textwidth]{logo-sqldeveloper}
    \caption{Logo SQL Developer}
    \label{fig:figure13}
    \end{center}
\end{figure}